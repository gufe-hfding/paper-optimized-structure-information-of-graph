%\documentclass[conference]{IEEEtran}
\documentclass[10pt, conference, letterpaper]{IEEEtran}
\IEEEoverridecommandlockouts
% The preceding line is only needed to identify funding in the first footnote. If that is unneeded, please comment it out.
\usepackage{cite}
\usepackage{amsmath,amssymb,amsfonts}
%\usepackage{algorithmic}
\usepackage{algorithm}  
\usepackage{algpseudocode} 
\usepackage{graphicx}
\usepackage{textcomp}
\usepackage{xcolor}
\newtheorem{theorem}{Theorem}
\newtheorem{lemma}{Lemma}
\newtheorem{proof}{Proof}[section]
\newtheorem{definition}{Definition}
\renewcommand{\algorithmicrequire}{\textbf{Input:}}
\renewcommand{\algorithmicensure}{\textbf{Output:}}
\def\BibTeX{{\rm B\kern-.05em{\sc i\kern-.025em b}\kern-.08em
    T\kern-.1667em\lower.7ex\hbox{E}\kern-.125emX}}
\begin{document}

\title{Minimized and Optimized Structural Information and Complexity Measurement of Network
	\thanks{This work is supported by National Natural Science Foundation of China (U1836205, 61662009, 61772008, 11761020); Science and Technology Program Foundation of Guizhou Province (Guizhou-Science-Contract-Major-Program [2018]3001, Guizhou-Science-Contract-Major-Program [2018]3007, Guizhou-Science-Contract-Major-Program [2017]3002, Guizhou-Science-Contract-Support [2019]2004, Guizhou-Science-Contract-Support [2018]2162, and Guizhou-Science-Contract-Foundation [2019]1049); The 13th Five-Year National Cryptography Development Foundation (MMJJ20170129); The Graduate Innovation Foundation of Guizhou University (Graduate-Science-Engineering 2016068)}
}

\author{%
	\IEEEauthorblockN{
		Hongfa Ding \IEEEauthorrefmark{1}\IEEEauthorrefmark{2}, 
		Changgen Peng\IEEEauthorrefmark{3}\IEEEauthorrefmark{4},
		Youliang Tian\IEEEauthorrefmark{3}\IEEEauthorrefmark{4},    Shuwen Xiang \IEEEauthorrefmark{1}
	}
	
	\IEEEauthorblockA{
		\IEEEauthorrefmark{1}College of Mathematics and statistics, Guizhou University, China.
		\\\{hongfa.ding\}@foxmail.com, \{shwxiang\}@vip.163.com
	} 
	
	\IEEEauthorblockA{
		\IEEEauthorrefmark{2}College of Information, Guizhou University of Finance and Economics, China.
		\\}
	\IEEEauthorblockA{
		\IEEEauthorrefmark{3}State key Laboratory of Public Big Data, Guizhou University, China.
		\\\{peng\_stud, youliangtian\}@163.com
	} 
	\IEEEauthorblockA{
		\IEEEauthorrefmark{4}College of Computer Science and Technology, Guizhou University, China.
	} 
}

\maketitle

\begin{abstract}
This document is a model and instructions for \LaTeX.
This and the IEEEtran.cls file define the components of your paper [title, text, heads, etc.]. *CRITICAL: Do Not Use Symbols, Special Characters, Footnotes, 
or Math in Paper Title or Abstract.
\end{abstract}

\begin{IEEEkeywords}
component, formatting, style, styling, insert
\end{IEEEkeywords}

\section{Introduction}

Challenges:

1. There is no such definition of optimal coding tree of structural information similar to that of a distribution $P=(p_1, p_2, \cdots, p_n)$. 

2. There is no definition of minimized and optimal structural information.

3. There is no simple and explicit algorithm to estimate the structural information defined by Li and Pan \cite{li2016}. i.e. How to estimate Two-Dimensional Structural Information and K-Dimensional Structural Information directly rather than traversing all the partitions of underlying graph.

4. And there is no algorithm to estimate the optimal structural information explicitly and directly rather than estimate all the $k$-Dimensional Structural Information where $1 \leq k \leq n$.


\section{Preliminaries}\label{sec:preliminaries}

In this section, we first present some definitions and notations of graph, then give the definition of $k$-core and degeneracy, and the algorithm to extract $k$-core decomposition. At last, basic theory of structural information is given.
\subsection{Notations of Graph}\label{subsec:notations}




\subsection{Degeneracy and $k$-core Decomposition}
\label{subsec:k-core}
Let $G$ be a graph and $G'$ be a subgraph of $G$ induced by a set of vertices $S$. Then $G'$ is called a $k$-core $C_k$ of $G$, if it a maximal subgraph of $G$, and each vertex's degree $d_{G'}(v) \geq k$. Each $k$-core subgraph of $G$ is unique, and not necessarily connected.
\subsection{Structural information}\label{subsec:structural_information}
All the definitions about structural information are suggested by Li and Pan \cite{li2016},
\begin{definition}
	\label{def:structural_information}
	(Structural Information of a Graph by a Partitioning Tree): For a partitioning tree $\mathcal{T}$ of an undirected and connected graph $G=(V,E)$, thus the structural information of $G$ by $\mathcal{T}$ is:
	
	\begin{enumerate}
		\item For each node $\alpha \in \mathcal{T}/\lambda$, where $\lambda$ is the root of $\mathcal{T}$, then define the structural information of $\alpha$ is:
		
		\begin{equation}
		\label{eq:information_alpha}
		H^{\mathcal{T}}(G;\alpha)=-\frac{g_{\alpha}}{2m}log_2\frac{V_{\alpha}}{V_{\alpha^-}}
		\end{equation}
		where $g_{\alpha}$ is the number of edges from nodes in $T_{\alpha}$ to nodes outside $T_{\alpha}$, $V_{\alpha}$ is the volume of set $T_{\alpha}$. 
		
		\item The structural information of $G$ is:
		\begin{equation}
		\label{eq:information_alpha}
			\mathcal{H}^{\mathcal{T}}(G)=\sum_{\alpha \in \mathcal{T}/\lambda}H^{\mathcal{T}}(G;\alpha)
		\end{equation}		
	\end{enumerate}
\end{definition}

\begin{definition}\label{def:kdimensional_information}
($K$-Dimensional Structural Information): Let $G=(V,E)$ is an undirect and connected graph, then $K$-dimensional structural information of $G$ is defined as:
\begin{equation}
\label{eq:kdimensional_information}
\mathcal{H}^{K}(G)=\underset{\mathcal{T}}{min}\{\mathcal{H}^{\mathcal{T}}(G)\} 
\end{equation}	
where $\mathcal{T}$ ranges over all of the partitioning trees of $G$ of height $K$.	
\end{definition}


\section{Optimal Structural Information}
\label{sec:optimal_strctrual_information}

\subsection{core-based Graph Partition}

\begin{definition}\label{def:unconnected_partition}
	(Unconnected based Graph Partition): Let $G=(V,E)$ be a undirect and unconnected graph, and there are $t$ connected sub-graphs $G_1, G_2,\cdots, G_t$ in $G$. Then the unconnected based graph Partition can be defined as
	\begin{equation}
	P(G)=\{G_1, G_2,\cdots, G_t\}
	\end{equation} 
\end{definition}

For example, in Fig. ...

\begin{definition}
	(Core-based Graph Partition): Let $G=(V,E)$ be a undirect and connected graph. Then the core based graph partition can be defined as
	\begin{equation}
	\label{eq:core-based-partition}
	P^{core}(G)=\{P(C_{\delta^*}),P(C_{\delta^*}-C_{\delta^*-1}),\cdots, P(C_{2}-C_{1})\}
	\end{equation} 
	where $\delta^*$ is degeneracy of $G$, $C_{\delta^*},C_{\delta^*-1},\cdots,C_{2},C_{1}$ are the $\delta^*$-core, $(\delta^*-1)$-core, ...,  $2$-core, $1$-core sub-graphs of $G$, respectively. And $P(\cdot)$ is a partition of graph $\cdot$ by the unconnected sub-graph(s) by following Def. \ref{def:unconnected_partition}.
\end{definition}

For example, in Fig. ...


\begin{theorem}
	Let $G=(V,E)$ be a undirect and connected graph, and $P^{core}(G)$ is the core based graph partition of $G$. Then $P^{core}(G)$ is the optimal partition of $G$ by the minimized 2-dimensional structural entropy of $G$.
\end{theorem}

\begin{algorithm}  
	\caption{Partitioning $G$ by the core based graph partition}  
	\begin{algorithmic}[1] %每行显示行号  
		\Require An undirect and connected graph $G=(V,E)$, with $n$ vertexes and $m$ edges.
		\Ensure Core based graph partition $P^{core}(G)$.
		\State Let $C_1, C_2, \cdots, C_{\delta^*}$ be the 1-core, 2-core, $\cdots$, $\delta^*$-core of $G$.
		\State $P^{core}(G)= P(C_{\delta^*})$
		\For{each $i \in [1,\delta^*-1]$}
		\State $P^{core}(G)=P^{core}(G) \cup P(C_i-C_{i+1})$
		\EndFor
	\end{algorithmic} 
	\Return $P^{core}(G)$
\end{algorithm} 

\subsection{2-dimensional and Optimal Structural Information}
\begin{definition}\label{def:2dimensional_information}
	(Two-dimensional Structural Information): Let $G=(V,E)$ be a undirect and connected graph, and $P^{core}(G)=\{X_1,X_2,\cdots,X_L\}$ be the core based graph partition of $G$. Then the two-dimensional structural information of $G$ can be defined as
	\begin{equation}
	\mathcal{H}^{2}(G)=\sum_{j=1}^{L}\frac{V_j}{2m}\cdot H(\frac{d_{1}^{j}}{V_j},\cdots,\frac{d_{n_j}^{j}}{V_j})-\sum_{j=1}^{L}\frac{g_j}{2m}log_2 \frac{V_j}{2m}
	\end{equation} 
\end{definition}

Def. \ref{def:2dimensional_information} is a variant of Def. 9 in \cite{li2016}. However, by Def. \ref{def:2dimensional_information}, we can estimate the two-dimensional structural information directly with an extremely lower time complexity. Here we suggest an algorithm to estimate two-dimensional structural information of $G$ as Alg. \ref{alg:2dimensional_information}.

\begin{algorithm}  
	\caption{Estimating 2-dimensional structural information of graph}\label{alg:2dimensional_information}  
	\begin{algorithmic}[1] %每行显示行号  
		\Require An undirect and connected graph $G=(V,E)$, with $n$ vertices and $m$ edges.
		\Ensure 2-dimensional structural information $\mathcal{H}^{2}(G)$.
		\State $P^{core}(G)= \{X_1,X_2,\cdots,X_L\}$
		\State $\mathcal{H}^{2}(G)=0$
		\For{each $j \in [1,L]$}
		\State $n_j=|X_j|$
		\State $H(X_j)=-\sum_{i=1}^{n_j}\frac{d_i^{(j)}}{V_j}log_2 \frac{d_i^{(j)}}{V_j}$
		\State $\mathcal{H}^{2}(G)+=\frac{V_j}{2m}H(X_j)-\frac{g_j}{2m}log_2\frac{V_j}{2m}$
		\EndFor
	\end{algorithmic} 
	\Return $\mathcal{H}^{2}(G)$
\end{algorithm} 

\begin{definition}
 (Optimal Structural Information): Let $G=(V,E)$ is an undirect and connected graph, then optimal structural information of $G$ is defined as:
 \begin{equation}
 \label{eq:optimal_information}
 \mathcal{H}^{Optimal}(G)=\underset{K\in [1,n]}{min}\{\mathcal{H}^{K}(G)\}
 \end{equation} 
 \end{definition}

\begin{definition}
	(Normalized Optimal Structural Information): Let $G=(V,E)$ is an undirect and connected graph, then normalized optimal structural information of $G$ is defined as:
	\begin{equation}
	\label{eq:normalized_optimal_information}
	\mathcal{H}^{Normalized}(G)=\frac{\mathcal{H}^{Optimal}(G)}{\mathcal{H}^{1}(G)}
	\end{equation} 
\end{definition}

\subsection{Characterization}


\section{Estimation Algorithm of Optimal Structural Information}


\begin{itemize}
\item Use either SI (MKS) or CGS as primary units. (SI units are encouraged.) English units may be used as secondary units (in parentheses). An exception would be the use of English units as identifiers in trade, such as ``3.5-inch disk drive''.
\item Avoid combining SI and CGS units, such as current in amperes and magnetic field in oersteds. This often leads to confusion because equations do not balance dimensionally. If you must use mixed units, clearly state the units for each quantity that you use in an equation.
\item Do not mix complete spellings and abbreviations of units: ``Wb/m\textsuperscript{2}'' or ``webers per square meter'', not ``webers/m\textsuperscript{2}''. Spell out units when they appear in text: ``. . . a few henries'', not ``. . . a few H''.
\item Use a zero before decimal points: ``0.25'', not ``.25''. Use ``cm\textsuperscript{3}'', not ``cc''.)
\end{itemize}

\subsection{Equations}
Number equations consecutively. To make your 
equations more compact, you may use the solidus (~/~), the exp function, or 
appropriate exponents. Italicize Roman symbols for quantities and variables, 
but not Greek symbols. Use a long dash rather than a hyphen for a minus 
sign. Punctuate equations with commas or periods when they are part of a 
sentence, as in:
\begin{equation}
a+b=\gamma\label{eq}
\end{equation}

Be sure that the 
symbols in your equation have been defined before or immediately following 
the equation. Use ``\eqref{eq}'', not ``Eq.~\eqref{eq}'' or ``equation \eqref{eq}'', except at 
the beginning of a sentence: ``Equation \eqref{eq} is . . .''

\subsection{\LaTeX-Specific Advice}

Please use ``soft'' (e.g., \verb|\eqref{Eq}|) cross references instead
of ``hard'' references (e.g., \verb|(1)|). That will make it possible
to combine sections, add equations, or change the order of figures or
citations without having to go through the file line by line.

Please don't use the \verb|{eqnarray}| equation environment. Use
\verb|{align}| or \verb|{IEEEeqnarray}| instead. The \verb|{eqnarray}|
environment leaves unsightly spaces around relation symbols.

Please note that the \verb|{subequations}| environment in {\LaTeX}
will increment the main equation counter even when there are no
equation numbers displayed. If you forget that, you might write an
article in which the equation numbers skip from (17) to (20), causing
the copy editors to wonder if you've discovered a new method of
counting.

{\BibTeX} does not work by magic. It doesn't get the bibliographic
data from thin air but from .bib files. If you use {\BibTeX} to produce a
bibliography you must send the .bib files. 

{\LaTeX} can't read your mind. If you assign the same label to a
subsubsection and a table, you might find that Table I has been cross
referenced as Table IV-B3. 

{\LaTeX} does not have precognitive abilities. If you put a
\verb|\label| command before the command that updates the counter it's
supposed to be using, the label will pick up the last counter to be
cross referenced instead. In particular, a \verb|\label| command
should not go before the caption of a figure or a table.

Do not use \verb|\nonumber| inside the \verb|{array}| environment. It
will not stop equation numbers inside \verb|{array}| (there won't be
any anyway) and it might stop a wanted equation number in the
surrounding equation.

\subsection{Some Common Mistakes}\label{SCM}
\begin{itemize}
\item The word ``data'' is plural, not singular.
\item The subscript for the permeability of vacuum $\mu_{0}$, and other common scientific constants, is zero with subscript formatting, not a lowercase letter ``o''.
\item In American English, commas, semicolons, periods, question and exclamation marks are located within quotation marks only when a complete thought or name is cited, such as a title or full quotation. When quotation marks are used, instead of a bold or italic typeface, to highlight a word or phrase, punctuation should appear outside of the quotation marks. A parenthetical phrase or statement at the end of a sentence is punctuated outside of the closing parenthesis (like this). (A parenthetical sentence is punctuated within the parentheses.)
\item A graph within a graph is an ``inset'', not an ``insert''. The word alternatively is preferred to the word ``alternately'' (unless you really mean something that alternates).
\item Do not use the word ``essentially'' to mean ``approximately'' or ``effectively''.
\item In your paper title, if the words ``that uses'' can accurately replace the word ``using'', capitalize the ``u''; if not, keep using lower-cased.
\item Be aware of the different meanings of the homophones ``affect'' and ``effect'', ``complement'' and ``compliment'', ``discreet'' and ``discrete'', ``principal'' and ``principle''.
\item Do not confuse ``imply'' and ``infer''.
\item The prefix ``non'' is not a word; it should be joined to the word it modifies, usually without a hyphen.
\item There is no period after the ``et'' in the Latin abbreviation ``et al.''.
\item The abbreviation ``i.e.'' means ``that is'', and the abbreviation ``e.g.'' means ``for example''.
\end{itemize}
An excellent style manual for science writers is \cite{b7}.

\subsection{Authors and Affiliations}
\textbf{The class file is designed for, but not limited to, six authors.} A 
minimum of one author is required for all conference articles. Author names 
should be listed starting from left to right and then moving down to the 
next line. This is the author sequence that will be used in future citations 
and by indexing services. Names should not be listed in columns nor group by 
affiliation. Please keep your affiliations as succinct as possible (for 
example, do not differentiate among departments of the same organization).

\subsection{Identify the Headings}
Headings, or heads, are organizational devices that guide the reader through 
your paper. There are two types: component heads and text heads.

Component heads identify the different components of your paper and are not 
topically subordinate to each other. Examples include Acknowledgments and 
References and, for these, the correct style to use is ``Heading 5''. Use 
``figure caption'' for your Figure captions, and ``table head'' for your 
table title. Run-in heads, such as ``Abstract'', will require you to apply a 
style (in this case, italic) in addition to the style provided by the drop 
down menu to differentiate the head from the text.

Text heads organize the topics on a relational, hierarchical basis. For 
example, the paper title is the primary text head because all subsequent 
material relates and elaborates on this one topic. If there are two or more 
sub-topics, the next level head (uppercase Roman numerals) should be used 
and, conversely, if there are not at least two sub-topics, then no subheads 
should be introduced.

\subsection{Figures and Tables}
\paragraph{Positioning Figures and Tables} Place figures and tables at the top and 
bottom of columns. Avoid placing them in the middle of columns. Large 
figures and tables may span across both columns. Figure captions should be 
below the figures; table heads should appear above the tables. Insert 
figures and tables after they are cited in the text. Use the abbreviation 
``Fig.~\ref{fig}'', even at the beginning of a sentence.

\begin{table}[htbp]
\caption{Table Type Styles}
\begin{center}
\begin{tabular}{|c|c|c|c|}
\hline
\textbf{Table}&\multicolumn{3}{|c|}{\textbf{Table Column Head}} \\
\cline{2-4} 
\textbf{Head} & \textbf{\textit{Table column subhead}}& \textbf{\textit{Subhead}}& \textbf{\textit{Subhead}} \\
\hline
copy& More table copy$^{\mathrm{a}}$& &  \\
\hline
\multicolumn{4}{l}{$^{\mathrm{a}}$Sample of a Table footnote.}
\end{tabular}
\label{tab1}
\end{center}
\end{table}

\begin{figure}[htbp]
%\centerline{\includegraphics{fig1.png}}
\caption{Example of a figure caption.}
\label{fig}
\end{figure}

Figure Labels: Use 8 point Times New Roman for Figure labels. Use words 
rather than symbols or abbreviations when writing Figure axis labels to 
avoid confusing the reader. As an example, write the quantity 
``Magnetization'', or ``Magnetization, M'', not just ``M''. If including 
units in the label, present them within parentheses. Do not label axes only 
with units. In the example, write ``Magnetization (A/m)'' or ``Magnetization 
\{A[m(1)]\}'', not just ``A/m''. Do not label axes with a ratio of 
quantities and units. For example, write ``Temperature (K)'', not 
``Temperature/K''.

\section*{Acknowledgment}

The preferred spelling of the word ``acknowledgment'' in America is without 
an ``e'' after the ``g''. Avoid the stilted expression ``one of us (R. B. 
G.) thanks $\ldots$''. Instead, try ``R. B. G. thanks$\ldots$''. Put sponsor 
acknowledgments in the unnumbered footnote on the first page.

\section*{References}

Please number citations consecutively within brackets \cite{b1}. The 
sentence punctuation follows the bracket \cite{b2}. Refer simply to the reference 
number, as in \cite{b3}---do not use ``Ref. \cite{b3}'' or ``reference \cite{b3}'' except at 
the beginning of a sentence: ``Reference \cite{b3} was the first $\ldots$''

Number footnotes separately in superscripts. Place the actual footnote at 
the bottom of the column in which it was cited. Do not put footnotes in the 
abstract or reference list. Use letters for table footnotes.

Unless there are six authors or more give all authors' names; do not use 
``et al.''. Papers that have not been published, even if they have been 
submitted for publication, should be cited as ``unpublished'' \cite{b4}. Papers 
that have been accepted for publication should be cited as ``in press'' \cite{b5}. 
Capitalize only the first word in a paper title, except for proper nouns and 
element symbols.

For papers published in translation journals, please give the English 
citation first, followed by the original foreign-language citation \cite{b6}.

\bibliographystyle{IEEEtranS}
\bibliography{references}

\end{document}
